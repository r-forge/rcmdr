\documentclass{article}%
\usepackage[round]{natbib}   % For in-text citations
\bibliographystyle{apa} % For bibliography style
\usepackage{amsmath}%
\setcounter{MaxMatrixCols}{30}%
\usepackage{amsfonts}%
\usepackage{amssymb}%
\usepackage{graphicx}

\setlength{\oddsidemargin}{0in}
\setlength{\evensidemargin}{0in}
\setlength{\topmargin}{-0.5in}
\setlength{\textwidth}{6.5in}
\setlength{\textheight}{9in}

\begin{document}

\title{Getting Started With the R Commander\thanks{Parts of this manual are adapted
and updated from Fox (2005). Please address correspondence to
\texttt{jfox@mcmaster.ca}. }}
\author{John Fox and Milan Bouchet-Valat}
\date{Version 2.2-1 (last modified: 25 August 2015)}
\maketitle

\section{Introduction}

The\textbf{\ R Commander} \citep{Fox05} provides a graphical user
interface (\textquotedblleft GUI\textquotedblright) to the open-source
\textbf{R } statistical computing environment \citep{RCore15}. \textbf{R} is a command-driven system, and new users
often find learning \textbf{R} challenging. This is particularly true of those
who are new to statistical methods, such as students in basic-statistics
courses. By providing a point-and-click interface to \textbf{R}, the \textbf{R
Commander} allows these users to focus on statistical methods rather than on
remembering and formulating \textbf{R} commands. Moreover, by rendering the
generated commands visible to users, the \textbf{R Commander} has the
potential for easing the transition to writing \textbf{R} commands, at least
for some users. The \textbf{R Commander}, however, accesses only a small
fraction of the capabilities of \textbf{R} and the literally thousands of
\textbf{R} packages contributed by users to the Comprehensive R Archive
Network (CRAN). The \textbf{R Commander} is itself extensible through plug-in
packages, and many such plug-ins are now available on CRAN (see Section
\ref{sec-plugins} of this document).

This document directly describes the use of the \textbf{R Commander} under the
\textbf{Windows} version of \textbf{R}. There are small differences in the
appearance and use of the \textbf{R Commander} under \textbf{Mac OS X} and on
\textbf{Linux} and \textbf{Unix} systems. Information about installing the
\textbf{R Commander} on these platforms is available by following the link to
the installation notes at the \textbf{R Commander} web page
$<$\texttt{http://socserv.socsci.mcmaster.ca/ jfox/Misc/Rcmdr/index.html}$>$ or at
$<$\texttt{tinyurl.com/Rcmdr}$>$.

We use the following typographical conventions in this document: Names of
software, such as \textbf{Windows}, \textbf{R}, the \textbf{Rcmdr} package,
and the \textbf{R Commander}, are set in \textbf{boldface} type. The names of
GUI elements such as menus, menu \ items, windows, and dialog boxes, are set
in \emph{italic} type. Variable names, names of data sets, and \textbf{R}
commands are set in a \texttt{typewriter font}.

\section{Starting the R Commander}

Once \textbf{R} is running, simply loading the \textbf{Rcmdr} package by
typing the command \texttt{library(Rcmdr)} into the \emph{R Console} starts
the \textbf{R Commander} graphical user interface. To function optimally under
\textbf{Windows}, the \textbf{R Commander} prefers the single-document
interface (\textquotedblleft SDI\textquotedblright) to \textbf{R}%
.\footnote{The \textbf{Windows} version of \textbf{R} is normally run from a
multiple-document interface (\textquotedblleft MDI\textquotedblright), which
contains the \emph{R Console} window, \emph{Graphical Device} windows created
during the session, and other windows related to the \textbf{R} process. In
contrast, under the single-document interface (\textquotedblleft
SDI\textquotedblright), the \emph{R Console} and \emph{Graphical Device}
windows are not contained within a master window. There are several ways to
run \textbf{R} in SDI mode --- for example, by selecting the SDI when
\textbf{R} is installed, by editing the \texttt{Rconsole} file in \textbf{R}'s
\texttt{etc} subdirectory, or by adding \texttt{-{}-sdi} to the \emph{Target}
field in the \emph{Shortcut} tab of the \textbf{R} desktop icon's
\emph{Properties}. You should be able to use the \textbf{R Commander} with the
MDI, but it will not appear within the master \textbf{R} window and arranging
the screen will be inconvenient.} After loading the package, \emph{R Console}
and \emph{R Commander} windows should appear more or less as in Figures
\ref{fig-1} and \ref{fig-1-b}.\footnote{Most of the ``screen-shots'' in this document
were produced with an earlier version of \textbf{R} and the \textbf{R Commander}.
Screen images were only updated when their appearance or content has changed.}
These and other screen images in this document
were created under \textbf{Windows 7}; if you use another version of
\textbf{Windows} (or, of course, another computing platform), then the
appearance of the screen may differ.\footnote{Notice that the \textbf{R
Commander} requires some packages in addition to several of the
\textquotedblleft recommended\textquotedblright\ packages that are normally
distributed with \textbf{R}. The \textbf{Rcmdr} package, the required
packages, and many other contributed packages are available for download from
the Comprehensive R Archive Network (CRAN) at
$<$\textsf{http://cran.r-project.org/}$>$%.
\par
If these packages are not installed, the \textbf{R Commander} will offer to
install them from the Internet or from local files (e.g., on a CD/ROM or USB flash drive). If you
install the \textbf{Rcmdr} package via the \textbf{Windows} \textquotedblleft
R GUI,\textquotedblright\ not all of the packages on which the \textbf{Rcmdr}
package depends will be installed. You can install the \textbf{Rcmdr} package
and all of the packages on which it depends via the \texttt{install.packages}
function, setting the argument \texttt{dependencies = TRUE}, but because of
recursive dependencies, that will install more packages than are strictly
necessary for the \textbf{R Commander} to function.
\par
Thanks to Dirk Eddelbuettel, \textbf{Debian Linux} users need only issue the
command \texttt{\$ apt-get install r-cran-rcmdr} to install the \textbf{Rcmdr}
package along with all of the packages that it requires. In any event,
building and installing the \textbf{Rcmdr} package on \textbf{Linux} systems
is typically straightforward. The task is a little more complicated under
\textbf{Mac OS X}, since the \textbf{tcltk} package on which the
\textbf{Rcmdr} depends requires that \textbf{X-Windows }be installed --- see
the \textbf{R Commander} installation notes.}%

\begin{figure}[ptb]%
\centering
\includegraphics[
natheight=6.560400in,
natwidth=9.080100in,
height=3.4388in,
width=4.7497in
]%
{fig1.jpg}%
\caption{The \emph{R Console} window after loading the \textbf{Rcmdr}
package.}%
\label{fig-1}%
\end{figure}

\begin{figure}[ptb]%
\centering
\includegraphics[
natheight=10.800300in,
natwidth=9.293400in,
height=5.6272in,
width=4.846in
]%
{fig-1-b.jpg}%
\caption{The \emph{R Commander} window at start-up.}%
\label{fig-1-b}%
\end{figure}

The \emph{R Commander} and \emph{R Console} windows float freely on the
desktop. You will normally use the \textbf{R Commander}'s menus and dialog
boxes to read, manipulate, and analyze data, and you can safely minimize the
\emph{R Console} window.

\begin{itemize}
\item \textbf{R} commands generated by the \textbf{R Commander} GUI appear in
the \emph{R Script} tab in the upper pane of the main \emph{R Commander}
window. You can also type \textbf{R} commands directly into the script
pane;\footnote{You can also type commands at the \texttt{$>$} (greater-than) prompt
in the \emph{R Console}, but output generated by these
commands will not appear in the \textbf{R Commander} \emph{Output} pane and
error and warning messages will normally not be visible.} the main purpose of
the \textbf{R Commander}, however, is to avoid having to type commands. The
second tab in the upper pane (labelled \emph{R Markdown}) also accumulates the
commands produced by the\textbf{\ R Commander} and can be used to generate
printed reports; the \emph{R Markdown} tab is described in Section
\ref{sec-markdown}.

\item Printed output appears by default in the second pane (labelled
\emph{Output}).

\item The lower, gray pane (labelled \emph{Messages}) displays error messages,
warnings, and some other information (\textquotedblleft
notes\textquotedblright), such as the start-up message in Figure \ref{fig-1-b}.

\item When you create graphs, these will appear in a separate \emph{Graphics
Device} window.
\end{itemize}

There are several menus along the top of the \emph{R Commander} window:

\begin{description}
\item[File] Menu items for loading and saving script files; for saving output
and the \textbf{R} workspace; and for exiting.

\item[Edit] Menu items (\emph{Cut}, \emph{Copy}, \emph{Paste}, etc.) for
editing text in the various panes and tabs. Right-clicking in one of these
panes or tabs also brings up an edit \textquotedblleft
context\textquotedblright\ menu.

\item[Data] Submenus containing menu items for reading and manipulating data.

\item[Statistics] Submenus containing menu items for a variety of statistical analyses.

\item[Graphs] Menu items for creating various statistical graphs.

\item[Models] Menu items and submenus for obtaining numerical summaries,
confidence intervals, hypothesis tests, diagnostics, and graphs for a
statistical model, and for adding diagnostic quantities, such as residuals, to
the data set.

\item[Distributions] Submenus to obtain cumulative probabilities, probability
densities or masses, quantiles, and graphs of standard statistical
distributions (to be used, for example, as a substitute for statistical
tables), and to generate samples from these distributions.

\item[Tools] Menu items for loading \textbf{R} packages unrelated to the
\textbf{Rcmdr} package (e.g., to access data saved in another package); for
loading \textbf{Rcmdr} plug-in packages (see \citealp{Fox07}, \citealp{FoxCarvalho12},
and Section~\ref{sec-plugins} below);
for setting most\textbf{\ R Commander} options and for
saving options so that they will be applied in subsequent sessions; and for
installing optional auxiliary software (see Section~\ref{sec-markdown}).

\item[Help] Menu items to obtain information about the \textbf{R Commander}
(including this manual) and associated software. As well, each \textbf{R
Commander} dialog box has a \emph{Help} button (see below).
\end{description}

The complete menu \textquotedblleft tree\textquotedblright\ for the \textbf{R
Commander} (version 2.2-0) is shown below. Most menu items lead to dialog
boxes, as illustrated later in this manual. Menu items are inactive
(\textquotedblleft grayed out\textquotedblright) if they are inapplicable to
the current context. For example, if a data set contains no factors
(categorical variables), the menu items for contingency tables will be
inactive.\footnote{Some menu items may not be displayed in certain
circumstances. For example, the \textbf{R Markdown} menu items in the
\emph{File} menu will be displayed only if the \emph{R Markdown} tab is
activated. The menu items shown here are those that are available by default
when the \textbf{R Commander} is running under \textbf{Windows}. The menus
also include dividers, which are not shown here, and menu items leading to
dialog boxes are, as is conventional, followed by \texttt{...}, also not
shown.}
\begin{verbatim}
File - Change working directory
    |- Open script file
    |- Save script
    |- Save script as
    |- Open R Markdown file
    |- Save R Markdown file
    |- Save R Markdown file as
    |- Save output
    |- Save output as
    |- Save R workspace
    |- Save R workspace as
    |- Exit - from Commander
           |- from Commander and R
Edit - Edit R Markdown document
    |- Edit knitr document
    |- Remove last Markdown command block
    |- Remove last knitr command block
    |- Cut
    |- Copy
    |- Paste
    |- Delete
    |- Find
    |- Select all
    |- Undo
    |- Redo
    |- Clear Window
Data - New data set
    |- Load data set
    |- Merge data sets
    |- Import data - from text file, clipboard, or URL
    |             |- from SPSS data set
    |             |- from SAS xport file
    |             |- from Minitab data set
    |             |- from STATA data set
    |             |- from Excel file
    |- Data in packages - List data sets in packages
    |                  |- Read data set from attached package
    |- Active data set - Select active data set
    |                 |- Refresh active data set
    |                 |- Help on active data set (if available)
    |                 |- Variables in active data set
    |                 |- Set case names
    |                 |- Subset active data set
    |                 |- Aggregate variables in active data set
    |                 |- Remove row(s) from active data set
    |                 |- Stack variables in active data set
    |                 |- Remove cases with missing data
    |                 |- Save active data set
    |                 |- Export active data set
    |- Manage variables in active data set - Recode variable
                                          |- Compute new variable
                                          |- Add observation numbers to data set
                                          |- Standardize variables
                                          |- Convert numeric variables to factors
                                          |- Bin numeric variable
                                          |- Reorder factor levels
                                          |- Drop unused factor levels
                                          |- Define contrasts for a factor
                                          |- Rename variables
                                          |- Delete variables from data set
Statistics - Summaries - Active data set
          |           |- Numerical summaries
          |           |- Frequency distributions
          |           |- Count missing observations
          |           |- Table of statistics
          |           |- Correlation matrix
          |           |- Correlation test
          |           |- Shapiro-Wilk test of normality
          |- Contingency Tables - Two-way table
          |                    |- Multi-way table
          |                    |- Enter and analyze two-way table
          |- Means - Single-sample t-test
          |       |- Independent-samples t-test
          |       |- Paired t-test
          |       |- One-way ANOVA
          |       |- Multi-way ANOVA
          |- Proportions - Single-sample proportion test
          |             |- Two-sample proportions test
          |- Variances - Two-variances F-test
          |           |- Bartlett's test
          |           |- Levene's test
          |- Nonparametric tests - Two-sample Wilcoxon test
          |                     |- Single-sample Wilcoxon test
          |                     |- Paired-samples Wilcoxon test
          |                     |- Kruskal-Wallis test
          |                     |- Friedman rank-sum test
          |- Dimensional analysis - Scale reliability
          |                      |- Principal-components analysis
          |                      |- Factor analysis
          |                      |- Confirmatory factor analysis
          |                      |- Cluster analysis - k-means cluster analysis
          |                                         |- Hierarchical cluster analysis
          |                                         |- Summarize hierarchical clustering
          |                                         |- Add hierarchical clustering to data set
          |- Fit models - Linear regression
                       |- Linear model
                       |- Generalized linear model
                       |- Multinomial logit model
                       |- Ordinal regression model
Graphs - Color palette
      |- Index plot
      |- Dot plot
      |- Histogram
      |- Density estimate
      |- Stem-and-leaf display
      |- Boxplot
      |- Quantile-comparison plot
      |- Scatterplot
      |- Scatterplot matrix
      |- Line graph
      |- XY conditioning plot
      |- Plot of means
      |- Strip chart
      |- Bar graph
      |- Pie chart
      |- 3D graph - 3D scatterplot
      |          |- Identify observations with mouse
      |          |- Save graph to file
      |- Save graph to file - as bitmap
                           |- as PDF/Postscript/EPS
                           |- 3D RGL graph
Models - Select active model
      |- Summarize model
      |- Add observation statistics to data
      |- Confidence intervals
      |- Akaike Information Criterion (AIC)
      |- Bayesian Information Criterion (BIC)
      |- Stepwise model selection
      |- Subset model selection
      |- Confidence intervals
      |- Hypothesis tests - ANOVA table
      |                  |- Compare two models
      |                  |- Linear hypothesis
      |- Numerical diagnostics - Variance-inflation factors
      |                       |- Breusch-Pagan test for heteroscedasticity
      |                       |- Durbin-Watson test for autocorrelation
      |                       |- RESET test for nonlinearity
      |                       |- Bonferroni outlier test
      |- Graphs - Basic diagnostic plots
               |- Residual quantile-comparison plot
               |- Component+residual plots
               |- Added-variable plots
               |- Influence plot
               |- Effect plots
Distributions - Set random number generator seed
             |- Continuous distributions - Normal distribution - Normal quantiles
             |            |                                   |- Normal probabilities
             |            |                                   |- Plot normal distribution
             |            |                                   |- Sample from normal distribution
             |            |- t distribution - t quantiles
             |            |                |- t probabilities
             |            |                |- Plot t distribution
             |            |                |- Sample from t distribution
             |            |- Chi-squared distribution - Chi-squared quantiles
             |            |                          |- Chi-squared probabilities
             |            |                          |- Plot chi-squared distribution
             |            |                          |- Sample from chi-squared distribution
             |            |- F distribution - F quantiles
             |            |                |- F probabilities
             |            |                |- Plot F distribution
             |            |                |- Sample from F distribution
             |            |- Exponential distribution - Exponential quantiles
             |            |                          |- Exponential probabilities
             |            |                          |- Plot exponential distribution
             |            |                          |- Sample from exponential distribution
             |            |- Uniform distribution - Uniform quantiles
             |            |                      |- Uniform probabilities
             |            |                      |- Plot uniform distribution
             |            |                      |- Sample from uniform distribution
             |            |- Beta distribution - Beta quantiles
             |            |                   |- Beta probabilities
             |            |                   |- Plot beta distribution
             |            |                   |- Sample from beta distribution
             |            |- Cauchy distribution - Cauchy quantiles
             |            |                     |- Cauchy probabilities
             |            |                     |- Plot Cauchy distribution
             |            |                     |- Sample from Cauchy distribution
             |            |- Logistic distribution - Logistic quantiles
             |            |                       |- Logistic probabilities
             |            |                       |- Plot logistic distribution
             |            |                       |- Sample from logistic distribution
             |            |- Lognormal distribution - Lognormal quantiles
             |            |                        |- Lognormal probabilities
             |            |                        |- Plot lognormal distribution
             |            |                        |- Sample from lognormal distribution
             |            |- Gamma distribution - Gamma quantiles
             |            |                    |- Gamma probabilities
             |            |                    |- Plot gamma distribution
             |            |                    |- Sample from gamma distribution
             |            |- Weibull distribution - Weibull quantiles
             |            |                      |- Weibull probabilities
             |            |                      |- Sample from Weibull distribution
             |            |- Gumbel distribution - Gumbel quantiles
             |                                  |- Gumbel probabilities
             |                                  |- Plot Gumbel distribution
             |                                  |- Sample from Gumbel distribution
             |- Discrete distributions - Binomial distribution - Binomial quantiles
                          |                       |- Binomial tail probabilities
                          |                       |- Binomial probabilities
                          |                       |- Plot binomial distribution
                          |                       |- Sample from binomial distribution
                          |- Poisson distribution - Poisson quantiles
                          |                     |- Poisson tail probabilities
                          |                     |- Poisson probabilities
                          |                     |- Plot Poisson distribution
                          |                     |- Sample from Poisson distribution
                          |- Geometric distribution - Geometric quantiles
                          |                       |- Geometric tail probabilities
                          |                       |- Geometric probabilities
                          |                       |- Plot geometric distribution
                          |                       |- Sample from geometric distribution
                          |- Hypergeometric distribution - Hypergeometric quantiles
                          |                     |- Hypergeometric tail probabilities
                          |                     |- Hypergeometric probabilities
                          |                     |- Plot hypergeometric distribution
                          |                     |- Sample from hypergeometric distribution
                          |- Negative binomial distribution - Negative binomial quantiles
                                                  |- Negative binomial tail probabilities
                                                  |- Negative binomial probabilities
                                                  |- Plot negative binomial distribution
                                                  |- Sample from negative binomial distribution
Tools - Load package(s)
     |- Load Rcmdr plug-in(s)
     |- Options
     |- Save Rcmdr options
     |- Manage OS X app nap for R.app [Mac OS X mavericks or newer only]
     |- Install [optional] auxiliary software [if not already installed]
Help - Commander help
    |- Introduction to the R Commander
    |- R Commander website
    |- About Rcmdr
    |- Help on active data set (if available)
    |- Start R help system
    |- R website
    |- Using R Markdown
\end{verbatim}

The \textbf{R Commander} interface includes a few elements in addition to
menus and dialogs:

\begin{itemize}
\item Below the menus is a \textquotedblleft toolbar\textquotedblright\ with a
row of buttons.

\begin{itemize}
\item The left-most (flat) button shows the name of the active data set.
Initially there is no active data set. If you press this button, you will be
able to choose among data sets currently in memory (if there is more than
one). Most of the menus and dialogs in the \textbf{R Commander} reference the
active data set. (The \emph{File}, \emph{Edit}, and \emph{Distributions} menus
are exceptions.)

\item Two buttons allow you to open the \textbf{R Commander} data editor to modify the
active data set or a viewer to examine it.\footnote{By default, the \textbf{R Commander} data editor
is used if the number of values (cells) in the data set is 10,000 or fewer; for larger data sets,
the standard \textbf{R} editor is used.} The data-set viewer can remain open
while other operations are performed.\footnote{The data viewer, provided by
the \texttt{showData} function from David Firth's \textbf{relimp}
package \citep{Firth11}, can be slow for data sets with large numbers of
variables. When the number of variables exceeds a threshold (initially set to
100), the less aesthetically pleasing \textbf{R} \texttt{View} command is used
instead to display the data set. To use \texttt{View} regardless of the number
of variables, set the threshold to 0. See the \textbf{R Commander} help for
details.}

\item A flat button indicates the name of the active statistical model --- a
linear model (such as a linear-regression model), a generalized linear model,
a multinomial logit model, or an ordinal regression model.\footnote{\textbf{R
Commander} plug-in packages (Fox, 2007; Fox and Carvalho, 2012) may provide
additional classes of models.} Initially there is no active model. If there is
more than one model in memory associated with the active data set, you can
choose among the models by pressing the button. The \textbf{R Commander}
synchronizes models and the data sets to which they are fit.
\end{itemize}

\item Immediately below the toolbar is a pane containing the \emph{R Script}
tab, a large scrollable text window. As mentioned, commands generated by the
\textbf{R Commander} are copied into this window. You can edit the text in the
\emph{Script} tab or even type your own \textbf{R} commands into the window.
Pressing the \emph{Submit} button, which is at the right below the
\emph{Script} tab (or, alternatively, the key combination \emph{Ctrl-r}%
,\footnote{That is, hold down the \emph{Ctrl} (or \emph{Control}) key and
simultaneously press the \emph{r} key.} for \textquotedblleft
run,\textquotedblright\ or\emph{\ Ctrl-Tab}), causes the line containing the
cursor to be submitted (or resubmitted) for execution. If several lines are
selected (e.g., by left-clicking and dragging the mouse over them), then
pressing \emph{Submit} will cause all of them to be executed. Commands entered
into the \emph{R Script} tab can extend over more than one line, but all lines
must be submitted simultaneously. The key combination \emph{Ctrl-a} selects
all of the text in the \emph{Script} tab, and \emph{Ctrl-s} brings up a dialog
box to save the contents of the tab. The \emph{R Markdown} tab is described in
Section \ref{sec-markdown}.

\item Below the \emph{R Script} and \emph{R Markdown} tabs is a pane
containing a large scrollable and editable text window for \emph{Output}.
Commands echoed to the \emph{Output} pane appear in red, the resulting output
in dark blue (as in the\ standard \textbf{Windows} \emph{R Console}).

\item At the bottom is a small gray pane for \emph{Messages}. Error messages
are displayed in red text, warnings in green, and other messages in dark blue.
Errors and warnings also provide an audible cue by ringing a bell.
\end{itemize}

As mentioned, once you have loaded the \textbf{Rcmdr} package, you can
minimize the \emph{R Console}. The \emph{R Commander} window can also be
resized or maximized in the normal manner. If you resize the \textbf{R
Commander}, the width of subsequent \textbf{R} output is automatically
adjusted to fit the \emph{Output} pane.

The \textbf{R Commander} is highly configurable: We have described the default
configuration here. Changes to the configuration can be made via the
\emph{Tools }$\longrightarrow$\emph{\ Options\ldots}\ menu, or --- more
extensively --- by setting \textbf{R Commander }options in \textbf{R}%
.\footnote{A menu item that terminates in ellipses (i.e., three dots, ...)
leads to a dialog box; this is a standard GUI convention. In this document,
$\longrightarrow$ represents selecting a menu item or submenu from a menu.}
See \emph{Help }$\longrightarrow$\emph{\ Commander help} for details.

\section{Data Input}

Most of the procedures in the \textbf{R Commander} assume that there is an
active data set.\footnote{Procedures selected under via the
\emph{Distributions} menu are exceptions, as is \emph{Enter and analyze
two-way table...} under the \emph{Statistics }$\longrightarrow$%
\emph{\ Contingency tables} menu.} If there are several data sets in memory,
you can choose among them, but only one is active. When the \textbf{R
Commander} starts up, there is no active data set.

The \textbf{R Commander} provides several ways to get data into \textbf{R}:

\begin{itemize}
\item Using the \textbf{R Commander} data editor, You can enter data directly
via \emph{Data }$\longrightarrow$\emph{\ New data set...}. This is a
reasonable choice only for a very small data set.

\item You can import data from a plain-text (\textquotedblleft
ascii\textquotedblright) file or the clipboard, over the Internet from a URL,
from another statistical package (\textbf{Minitab}, \textbf{SPSS},
\textbf{SAS}, or \textbf{Stata}), or from an \textbf{Excel} spreadsheet.

\item You can read a data set that is included in an \textbf{R} package,
either typing the name of the data set (if you know it), or selecting the data
set in a dialog box.
\end{itemize}

\subsection{Reading Data From a Text File}

For example, consider the data file \texttt{Nations.txt}.\footnote{This file
resides in the \texttt{etc} subdirectory of the \textbf{Rcmdr} package. The
data are for 1998 and are from the United Nations.} The first few lines of the
file are as follows:
\begin{verbatim}
TFR contraception infant.mortality GDP region
Afghanistan                 6.90    NA  154  2848   Asia
Albania                     2.60    NA   32   863   Europe
Algeria                     3.81    52   44  1531   Africa
American-Samoa              NA      NA   11  NA     Oceania
Andorra                     NA      NA  NA   NA     Europe
Angola                      6.69    NA  124   355   Africa
Antigua                     NA      53   24  6966   Americas
Argentina                   2.62    NA   22  8055   Americas
Armenia                     1.70    22   25   354   Europe
Australia                   1.89    76    6 20046   Oceania
. . .

\end{verbatim}

\begin{itemize}
\item The first line of the file contains variable names: \texttt{TFR} (the
total fertility rate, expressed as number of children per woman),
\texttt{contraception} (the rate of contraceptive use among married women, in
percent), \texttt{infant.mortality} (the infant-mortality rate per 1000 live
births), \texttt{GDP} (gross domestic product per capita, in U.S. dollars),
and \texttt{region}.

\item Subsequent lines contain the data values themselves, one line per
country. The data values are separated by \textquotedblleft white
space\textquotedblright\ --- one or more blanks or tabs. Although it is
helpful to make the data values line up vertically, it is not necessary to do
so. Notice that the data lines begin with the country names. Because we want
these to be the \textquotedblleft row names\textquotedblright\ for the data
set, there is no corresponding variable name: That is, there are five variable
names but six data values on each line, the first of which is alphabetic. When
this happens, the \textbf{R} \texttt{read.table} command will interpret the
first value on each line as the row name.

\item Some of the data values are missing. In \textbf{R}, it is most
convenient to use \texttt{NA} (representing \textquotedblleft not
available\textquotedblright) to encode missing data, as we have done here.

\item The variables \texttt{TFR}, \texttt{contraception},
\texttt{infant.mortality}, and \texttt{GDP} are numeric (quantitative)
variables; in contrast, \texttt{region} contains region names. When the data
are read, \textbf{R} will treat \texttt{region} as a \textquotedblleft
factor\textquotedblright\ --- that is, as a categorical variable. In most
contexts, the \textbf{R Commander} distinguishes between numerical variables
and factors, and will try to prevent you from doing unreasonable things, such
as computing the mean of a factor.
\end{itemize}

To read the \texttt{Nations.txt} data file into \textbf{R}, select \emph{Data
}$\longrightarrow$\emph{\ Import data }$\longrightarrow$\emph{\ from text
file, clipboard, or URL...} from the \emph{R Commander} menus. This operation
brings up a \emph{Read Text Data} dialog, as shown in Figure \ref{fig-2}. The
default name of the data set is \texttt{Dataset}. We have changed the name to
\texttt{Nations}.

\begin{figure}[ptb]%
\centering
\includegraphics[
natheight=7.053500in,
natwidth=7.546700in,
height=3.6961in,
width=3.9535in
]%
{fig-2.jpg}%
\caption{Reading data from a text file.}%
\label{fig-2}%
\end{figure}

\begin{figure}[ptb]%
\centering
\includegraphics[
natheight=9.599800in,
natwidth=13.159800in,
height=5.0203in,
width=6.8709in
]%
{fig-2b.jpg}%
\caption{Open-file dialog for reading a text data file.}%
\label{fig-2b}%
\end{figure}


Valid \textbf{R} names begin with an upper- or lower-case letter (or a period,
\texttt{.}) and consist entirely of letters, periods, underscores
(\texttt{\_}), and numerals (i.e., \texttt{0}--\texttt{9}); in particular, do
not include any embedded blanks in a data-set name. \textbf{R} is
case-sensitive, and so, for example, \texttt{nations}, \texttt{Nations}, and
\texttt{NATIONS} are distinguished, and could be used to represent different
data sets.

Clicking the \emph{OK} button in the \emph{Read Text Data} dialog brings up an
\emph{Open }file dialog, shown in Figure \ref{fig-2b}. Here we navigated to
and selected the file \texttt{Nations.txt}. Clicking the \emph{Open} button in
the dialog causes the data file to be read. Once the data file is read, it
becomes the active data set in the \textbf{R Commander}. As a consequence, in
Figure \ref{fig-3}, the name of the data set appears in the data set button
near the top left of the \emph{R Commander} window.

We next clicked the \emph{View data set} button to bring up the data viewer
window, also shown in Figure \ref{fig-3}. The commands to read and view the
\texttt{Nations} data set (the \textbf{R} \texttt{read.table} and
\texttt{showData} commands) appear in the \emph{R Script} tab and
\emph{Output} pane. As well, when the data set is read and becomes the active
data set, a note appears in the \emph{Messages} pane. The\textbf{\ R
Commander} also issued a \texttt{library} command to load the \textbf{relimp}
package, which was used to display the data set; here, as in general,
\textbf{R} packages are loaded automatically by the \textbf{R Commander} as
they are needed.%

\begin{figure}[ptb]%
\centering
\includegraphics[
natheight=7.946800in,
natwidth=15.106600in,
height=3.6048in,
width=6.826in
]%
{fig-3.jpg}%
\caption{Displaying the active data set.}%
\label{fig-3}%
\end{figure}

The \texttt{read.table} command creates an \textbf{R} \textquotedblleft data
frame,\textquotedblright\ which is an object containing a rectangular
cases-by-variables data set: The rows of the data set represent cases or
observations and the columns represent variables. Data sets in the \textbf{R
Commander} are \textbf{R} data frames.

\subsection{Entering Data Directly}

You can enter data directly into the
\textbf{R Commander} basic spreadsheet-like data editor. A simple
alternative, which we in fact prefer, is to save the data in a plain-text file
(be careful, if you create the data file with a word processor, to save it as
a plain-text or \textquotedblleft ascii\textquotedblright\ file), typically
with file type \texttt{.txt}, and then to read the file as in the preceding
section, via \emph{Data }$\longrightarrow$\emph{\ Import data }%
$\longrightarrow$\emph{\ from text file, clipboard, or URL...} . If your data
are already in a spreadsheet program, such as \textbf{Excel}, you can simply
export the data to a comma-separated-values text file (\texttt{.csv} file),
and read the file into the R Commander, being careful to specify the
field separator as a comma. Recall that you can also
read an \textbf{Excel} spreadsheet directly.

As an example of direct data input, we use a very small data set from Problem
2.44 in \citet{Moore00}:

\begin{itemize}
\item Select \emph{Data }$\longrightarrow$\emph{\ New data set...} from the
\emph{R Commander} menus. Optionally enter a name for the data set, such as
\texttt{Problem2.44}, in the resulting dialog box, and click the \emph{OK}
button. (Remember that \textbf{R} names cannot include embedded blanks.) This
will bring up a \emph{Data Editor} window with an empty data set.

\item Enter the data from the problem into the first two columns of the data
editor. Add a column by clicking the \emph{Add column} button in the data editor
toolbar, or by selecting \emph{Add column} from the \emph{Edit} menu. Similarly,
add rows to the data set by clicking the \emph{Add row} button repeatedly or via
the \emph{Edit} menu.

\item You can move from one cell to another by using the arrow keys on your
keyboard or by pointing with the mouse and left-clicking. Originally, the
variables are named \texttt{var1} and \texttt{var2}, and the data values are
all \texttt{NA} (i.e., missing). When you type a new
variable name, row name, or data value into a cell of the data editor, the new value replaces what was
previously there. If you double-click in a cell, then the cell becomes \texttt{NA}.
When you are finished entering the data, the data-editor window should look like Figure \ref{fig-4}.%

\item In this example, both variables are numeric. If you type any non-numeric values in a column
in the data editor (other than the value \texttt{NA}), then the column will define a factor
(categorical variable) in the new data set. Values that contain blanks must
be enclosed in single or double quotes (e.g., \texttt{'some postsecondary'}, \texttt{"less than HS"}).

\begin{figure}[ptb]%
\centering
\includegraphics[
height=2.5in,
]%
{fig-4.jpg}%
\caption{Data editor after the data are entered.}%
\label{fig-4}%
\end{figure}


\item Select \emph{File }$\longrightarrow$\emph{Exit and save} from the \emph{Data
Editor} menus or click the \emph{OK}
button. The data set that you entered is now the active data set in
the \textbf{R Commander}.
\end{itemize}

\subsection{Reading Data from a Package\label{sec-data-in-packages}}

Many \textbf{R} packages include data. Data sets in packages can be listed in
a pop-up window via \emph{Data }$\longrightarrow$\emph{\ Data in packages
}$\longrightarrow$\emph{\ List data sets in packages}, and can be read into
the \textbf{R Commander} via \emph{Data }$\longrightarrow$\emph{\ Data in
packages }$\longrightarrow$\emph{\ Read data set from an attached
package}.\footnote{Not all data in packages are data frames but only data
frames are suitable for use in the \textbf{R Commander}. If you try to read
data that are not a data frame, an error message will appear in the messages
window.} The resulting dialog box is shown in Figure \ref{fig-6-5}. If you
know the name of a data set in a package then you can enter its name directly;
otherwise double-clicking on the name of a package displays its data sets in
the right list box; and double-clicking on a data set name copies the name to
the data-set entry field in the dialog.\footnote{In general in the \textbf{R
Commander}, when it is necessary to copy an item from a list box to another
location in a dialog, a double-click is required.} Pressing a letter key in
the \emph{Data set} list box will scroll to the next data set whose name
begins with that letter. You can access additional \textbf{R} packages that
are installed in your package library by \emph{Tools }$\longrightarrow
$\emph{\ Load packages}.%


\begin{figure}[ptb]%
\centering
\includegraphics[
natheight=3.813200in,
natwidth=5.106700in,
height=2.0108in,
width=2.6833in
]%
{fig-6-5.jpg}%
\caption{Reading data from an attached package --- in this case the
\texttt{Prestige} data set from the \textbf{car} package.}%
\label{fig-6-5}%
\end{figure}


\section{Creating Numerical Summaries and Graphs}

Once there is an active data set, you can use the \textbf{R Commander} menus
to produce a variety of numerical summaries and graphs. We will describe just
a few basic examples here. A good GUI should be largely self-explanatory: We
hope that once you see how the \textbf{R Commander} works, you will have
little trouble using it, assisted perhaps by the on-line help files.

In the initial examples below, we assume that the active data set is the
\texttt{Nations} data set, read from a text file in the previous section. If
you typed in the five-observation data set from Moore (2000), or read in the
\texttt{Prestige}\ data set from the \textbf{car} package --- operations that
were also described in the previous section --- then one of these is the
active data set. Recall that you can change the active data set by clicking on
the flat button with the active data set's name near the top left of the
\emph{R Commander} window, choosing from among a list of data sets currently
resident in memory.

Selecting \emph{Statistics }$\longrightarrow$\emph{\ Summaries }%
$\longrightarrow$\emph{\ Active data set} produces the results shown in Figure
\ref{fig-7}. For each numerical variable in the data set (\texttt{TFR},
\texttt{contraception}, \texttt{infant.mortality}, and \texttt{GDP}),
\textbf{R} reports the minimum and maximum values, the first and third
quartiles, the median, and the mean, along with the number of missing values.
For the categorical variable \texttt{region}, we get the number of
observations at each \textquotedblleft level\textquotedblright\ of the factor.
Had the data set included more than ten variables, the \textbf{R Commander}
would have asked us whether we really want to proceed --- potentially
protecting us from producing unwanted voluminous output. This menu item is
unusual in that it directly invokes an \textbf{R} command rather than leading
to a dialog box, as is more typical of \textbf{R Commander} menus.%

\begin{figure}[ptb]%
\centering
\includegraphics[
natheight=10.800300in,
natwidth=9.293400in,
height=5.6446in,
width=4.8609in
]%
{fig-7.jpg}%
\caption{Getting variable summaries for the active data set.}%
\label{fig-7}%
\end{figure}

\begin{figure}[ptb]%
\centering
\includegraphics[
natheight=4.466600in,
natwidth=6.533000in,
height=2.352in,
width=3.4255in
]%
{fig-8.jpg}%
\caption{The \emph{Data} tab in the \emph{Numerical Summaries} dialog box.}%
\label{fig-8}%
\end{figure}

For example, selecting \emph{Statistics }$\longrightarrow$\emph{\ Summaries
}$\longrightarrow$\emph{\ Numerical summaries...} brings up the dialog box in
Figure \ref{fig-8}. Only numerical variables appear in the variable list in
this dialog; the factor \texttt{region} is excluded because it is not sensible
to compute numerical summaries like the mean and standard deviation for a
factor. We selected the variable \texttt{infant.mortality} by left-clicking on
it.\footnote{To select a single variable in a variable-list box, simply
left-click on its name. In some contexts, you will have to (or want to) select
more than one variable. In these cases, the usual \textbf{Windows} conventions
apply: Left-clicking on a variable selects it and de-selects any variables
that have previously been selected; \emph{Shift-left-click} extends the
selection; and \emph{Ctrl-left-click} toggles the selection for an individual
variable.} The \emph{Numerical Summaries} dialog box has two tabs: \emph{Data}
and \emph{Statistics}. Click on the \emph{Statistics} tab to select it, as
shown in Figure \ref{fig-8a}. In this case, We'll take all of the default
statistics selections. Clicking \emph{OK}, produces the following output (in
the \emph{Output} pane):
\begin{verbatim}
> numSummary(Nations[,"infant.mortality"], statistics=c("mean", "sd", "IQR",
+   "quantiles"), quantiles=c(0,.25,.5,.75,1))
     mean       sd IQR 0% 25% 50% 75% 100%   n NA
 43.47761 38.75604  54  2  12  30  66  169 201  6
\end{verbatim}%

\begin{figure}[ptb]%
\centering
\includegraphics[
natheight=4.466600in,
natwidth=6.533000in,
height=2.352in,
width=3.4255in
]%
{fig-8a.jpg}%
\caption{The \emph{Statistics} tab in the \emph{Numerical Summaries} dialog
box.}%
\label{fig-8a}%
\end{figure}

\noindent By default, the \textbf{R} command that is executed prints out the
mean, standard deviation (\texttt{sd}), and interquartile range (\texttt{IQR})
of the variable, along with quantiles (percentiles) corresponding to the
minimum, the first quartile, the median, the third quartile, and the maximum;
\texttt{n} is the number of valid obserations, and \texttt{NA} the number of
missing values.

As is typical of \textbf{R Commander} dialogs, the \emph{Numerical Summaries}
dialog box in Figure \ref{fig-8} includes \emph{Help}, \emph{Reset},
\emph{OK}, \emph{Cancel}, and \emph{Apply} buttons.\footnote{The order of the
buttons varies according to the operating system, and is different, for
example, in \textbf{Mac OS X} than in \textbf{Windows}.} The \emph{Help}
button leads to a help page (which appears in your web browser) either for the
dialog itself or (as here) for an \textbf{R} function that the dialog invokes.
The \emph{Reset} button, which is present in most \textbf{R Commander}
dialogs, resets the dialog to its original state; otherwise, the dialog
retains selections from a previous invocation. Dialog state is also reset when
the active data set changes. As demonstrated, the \emph{OK} button closes the
dialog and generates an \textbf{R} command. The \emph{Apply} button also
generates a command, but then reopens the dialog in its current state,
facilitating the application of several similar operations. If you make an
error in a dialog box --- for example, clicking \emph{OK} without choosing a
variable in the \emph{Numerical Summarie}s dialog --- an error message will
typically appear and the dialog box will reopen.%

\begin{figure}[ptb]%
\centering
\includegraphics[
natheight=6.400100in,
natwidth=7.813200in,
height=3.3574in,
width=4.0913in
]%
{fig-8-1.jpg}%
\caption{Selecting a grouping variable in the \emph{Groups} dialog box.}%
\label{fig-8-1}%
\end{figure}

\begin{figure}[ptb]%
\centering
\includegraphics[
natheight=4.466600in,
natwidth=6.533000in,
height=2.352in,
width=3.4255in
]%
{fig-8-2.jpg}%
\caption{The \emph{Numerical Summaries} dialog box after the grouping variable
\texttt{region} has been chosen and with two numeric variables selected.}%
\label{fig-8-2}%
\end{figure}

The \emph{Numerical Summaries} dialog box also makes provision for computing
summaries within groups defined by the levels of a factor. Clicking on the
\emph{Summarize by groups...} button in the \emph{Data} tab brings up the
\emph{Groups} dialog, as shown in Figure \ref{fig-8-1}. Because there is only
one factor in the \texttt{Nations} data set, only the variable \texttt{region}
appears in the variable list, and it is pre-selected; clicking \emph{OK}
changes the \emph{Summarize by groups...} button to \emph{Summarize by region}
(see Figure \ref{fig-8-2}). In this case, we have selected two numerical
variables to summarize, \texttt{GDP} and \texttt{infant.mortality}. Clicking
\emph{OK} produces the following results in the \emph{Output} pane:
\begin{verbatim}
> numSummary(Nations[,c("GDP", "infant.mortality")], groups=Nations$region,
+   statistics=c("mean", "sd", "IQR", "quantiles"), quantiles=c(0,.25,.5,.75,1))

Variable: GDP
              mean        sd      IQR  0%     25%    50%      75%  100%  n NA
Africa    1196.000  2089.614   795.50  36  209.00  389.5  1004.50 11854 54  1
Americas  5398.000  6083.311  5268.50 386 1749.25 2765.5  7017.75 26037 40  1
Asia      4505.051  6277.738  6062.50 122  345.00 1079.0  6407.50 22898 39  2
Europe   13698.909 13165.412 24582.25 271 1643.75 9222.5 26226.00 42416 44  1
Oceania   8732.600 11328.708 16409.25 654 1102.75 2348.5 17512.00 41718 20  5

Variable: infant.mortality
             mean        sd  IQR 0%   25%  50%    75% 100%  n NA
Africa   85.27273 35.188095 50.0  7 61.00 85.0 111.00  169 55  0
Americas 25.60000 17.439713 24.0  6 12.00 21.5  36.00   82 40  1
Asia     45.65854 32.980001 50.0  5 22.00 37.0  72.00  154 41  0
Europe   11.85366  7.122363 10.0  5  6.00  8.0  16.00   32 41  4
Oceania  27.79167 29.622229 26.5  2  9.25 20.0  35.75  135 24  1
\end{verbatim}

\noindent Several other \textbf{R Commander} dialogs allow you to select a
grouping variable in this manner.

Making graphs with the \textbf{R Commander} is also straightforward. For
example, selecting \emph{Graphs }$\longrightarrow$\emph{\ Histogram... }from
the \textbf{R Commander} menus produces the \emph{Histogram} dialog box in
Figure \ref{fig-9}. There are \emph{Data} and \emph{Options} tabs in this
dialog. We'll take all the default options (the \emph{Options} tab isn't
shown), and clicking on \texttt{infant.mortality} followed by \emph{OK}, opens
a \emph{Graphics Device} window with the histogram shown in Figure
\ref{fig-10}. If you make several graphs in a session, then only the most
recent normally appears in the \emph{Graphics Device} window.\footnote{On
\textbf{Windows}, you can recall previous graphs using the \emph{Page Up} and
\emph{Page Down} keys on your keyboard if you first turn on the graph history
feature of the \textbf{Windows R} graphics device, via History
$\longrightarrow$ Recording. This feature is available only on
\textbf{Windows} systems. Dynamic three-dimensional scatterplots created by
\emph{Graphs }$\longrightarrow$\emph{\ 3D graph }$\longrightarrow$\ \emph{3D
scatterplot...} appear in a special \emph{RGL device} window; likewise, effect
displays created for statistical models \citep{Fox03, FoxHong09} via \emph{Models }$\longrightarrow
$\emph{\ Graphs }$\longrightarrow$\emph{\ Effect plots} appear in individual
graphics-device windows.}%

\begin{figure}[ptb]%
\centering
\includegraphics[
natheight=3.933600in,
natwidth=6.533000in,
height=2.073in,
width=3.4255in
]%
{fig-9.jpg}%
\caption{The \emph{Histogram} dialog. }%
\label{fig-9}%
\end{figure}

\begin{figure}[ptb]%
\centering
\includegraphics[
natheight=9.733500in,
natwidth=9.173000in,
height=4.4076in,
width=4.1569in
]%
{fig-10.jpg}%
\caption{A graphics window containing the histogram for infant mortality in
the \texttt{Nations} data set.}%
\label{fig-10}%
\end{figure}

\section{Statistical Models}

Several kinds of statistical models can be fit\ in the \textbf{R Commander}
using menu items under \emph{Statistics }$\longrightarrow$\emph{\ Fit models}:
linear models (by both \emph{Linear regression} and \emph{Linear model}),
generalized linear models, multinomial logit models, and ordinal regression
models such as the proportional-odds model [the latter two from Venables and
Ripley's \textbf{nnet} and \textbf{MASS} packages,
respectively \citep{VenablesRipley02}]. Although the resulting dialog boxes
differ in certain details (for example, the generalized linear model dialog
makes provision for selecting a distributional family and corresponding link
function), they share a common general structure, as illustrated in the
\emph{Linear Model} dialog in Figure \ref{fig-10-5}.\footnote{An exception is
the \emph{Linear Regression} dialog in which the response variable and
explanatory variables are simply selected by name from list boxes containing
the numeric variables in the current data set.} Before selecting
\emph{Statistics} $\longrightarrow$\ \emph{Fit models} $\longrightarrow$
\emph{Linear Model}, we made \texttt{Prestige} the active data set by clicking
on the active data set button and selecting \texttt{Prestige} from the
resulting list. Recall that the \texttt{Prestige} data were read from the
\textbf{car} package in Section \ref{sec-data-in-packages}.%

\begin{figure}[ptb]%
\centering
\includegraphics[
natheight=5.667100in,
natwidth=9.333300in,
height=2.9747in,
width=4.8817in
]%
{fig-10-5.jpg}%
\caption{The \emph{Linear Model} dialog box, with \texttt{Prestige} from the
\textbf{car} package as the active data set.}%
\label{fig-10-5}%
\end{figure}

\begin{itemize}
\item Double-clicking on a variable in the variable-list box copies it to the
model formula --- to the left-hand side of the formula, if it is empty,
otherwise to the right-hand side (with a preceding + sign if the context
requires it). Factors (categorical variables --- here, \texttt{type}) are parenthetically
labelled as such in the variable list.\footnote{Some data frames contain
logical variables (with values \texttt{TRUE} and \texttt{FALSE}) and character
variables, with values that are text strings (such as \texttt{"male"} and
\texttt{"female"}). If such variables are present, the R Commander will treat
them as if they were factors. In most context, this will work properly.
Character data read from plain-text files will automatically be converted to
factors.} Entering a factor into the right-hand side of a statistical model
formula generates dummy-variable regressors.

\item The top row of buttons in the toolbar above the formula can be used to
enter operators and parentheses into the right-hand side of the formula.

\item The bottom row of buttons in the toolbar can be conveniently used to
enter regression-spline and polynomial terms into the model formula, with the
degrees of freedom for splines and the degree of polynomials controlled by the
spin-boxes to the right of the buttons (defaulting to 5 df and degree 2, respectively).

\item You can also type directly into the formula fields, and indeed may have
to do so, for example, to put a term such as \texttt{log(income)} into the
formula, as we've done here. Some information on \textbf{R} model formulas may
be obtained by pressing the \emph{Model formula help} button in the
linear-model dialog.

\item The name of the model, here \texttt{LinearModel.1}, is automatically
generated, but you can substitute any valid \textbf{R} name.

\item You can type an \textbf{R} expression into the box labelled \emph{Subset
expression}; if supplied, this is passed to the \texttt{subset} argument of
the \texttt{lm} function, and is used to fit the model to a subset of the
observations in the data set. One form of subset expression is a logical
expression that evaluates to \texttt{TRUE} or \texttt{FALSE} for each
observation, such as \texttt{type != "prof"} (which would select all
non-professional occupations from the \texttt{Prestige} data set).

\item Optionally selecting a weight variable in the \emph{Weights} drop-down
list produces a weighted-least-squares (WLS) regression.
\end{itemize}

Clicking the \emph{OK} button generates the following command and\ output, and
makes \texttt{LinearModel.1} the active model, with its name displayed in the
\emph{Model} button:
\begin{verbatim}
> LinearModel.1 <- lm(prestige ~ (education + log(income))*type,
+   data=Prestige)

> summary(LinearModel.1)

Call:
lm(formula = prestige ~ (education + log(income)) * type, data = Prestige)

Residuals:
    Min      1Q  Median      3Q     Max
-13.970  -4.124   1.206   3.829  18.059

Coefficients:
                          Estimate Std. Error t value Pr(>|t|)
(Intercept)              -120.0459    20.1576  -5.955 5.07e-08 ***
education                   2.3357     0.9277   2.518  0.01360 *
log(income)                15.9825     2.6059   6.133 2.32e-08 ***
type[T.prof]               85.1601    31.1810   2.731  0.00761 **
type[T.wc]                 30.2412    37.9788   0.796  0.42800
education:type[T.prof]      0.6974     1.2895   0.541  0.58998
education:type[T.wc]        3.6400     1.7589   2.069  0.04140 *
log(income):type[T.prof]   -9.4288     3.7751  -2.498  0.01434 *
log(income):type[T.wc]     -8.1556     4.4029  -1.852  0.06730 .
---
Signif. codes:  0 '***' 0.001 '**' 0.01 '*' 0.05 '.' 0.1 ' ' 1

Residual standard error: 6.409 on 89 degrees of freedom
  (4 observations deleted due to missingness)
Multiple R-squared:  0.871, Adjusted R-squared:  0.8595
F-statistic: 75.15 on 8 and 89 DF,  p-value: < 2.2e-16
\end{verbatim}

Operations on the active model may be selected from the \emph{Models} menu.
For example, \emph{Models }$\longrightarrow$\emph{\ Hypothesis tests
}$\longrightarrow$\emph{\ Anova table...}, followed by selecting the default
\textquotedblleft Type-II\textquotedblright\ tests, produces the following output:
\begin{verbatim}
> Anova(LinearModel.1, type="II")
Anova Table (Type II tests)

Response: prestige
                 Sum Sq Df F value    Pr(>F)
education        1209.3  1 29.4446 4.912e-07 ***
log(income)      1690.8  1 41.1670 6.589e-09 ***
type              469.1  2  5.7103  0.004642 **
education:type    178.8  2  2.1762  0.119474
log(income):type  290.3  2  3.5344  0.033338 *
Residuals        3655.4 89
---
Signif. codes:  0 '***' 0.001 '**' 0.01 '*' 0.05 '.' 0.1 ' ' 1
\end{verbatim}

\section{Odds and Ends}

\subsection{Producing Reports\label{sec-markdown}}

In its default configuration, the \textbf{R Commander} includes an \emph{R
Markdown} tab in the upper pane, which accumulates the commands generated
during the session in an\textbf{\ R Markdown}
document.\footnote{The\textbf{\ R Commander} can also optionally create a
\textbf{knitr} \textbf{LaTeX} document \citep{Xie13} and compile
it into a PDF file. This option requires a \textbf{LaTeX} installation, and is
activated by the setting the \textbf{Rcmdr} option \texttt{use.knitr} to
\texttt{TRUE}. see \emph{Help} $\longrightarrow$ \emph{Commander help} and
\emph{Tools} $\longrightarrow$ \emph{Options}.} Figure \ref{fig-11} shows
the\emph{\ R Markdown} tab for the current session, which we've scrolled to
the top of the generated document. As its name implies, \textbf{R Markdown} is
a simple markup language, which includes blocks of \textbf{R} commands, and
can be used to generate HTML (i.e., web) pages. For more information about
\textbf{R Markdown}, select \emph{Help} $\longrightarrow$\ \emph{Using R
Markdown} from the \textbf{R Commander} menus.\footnote{This and some other
items in the \emph{Help} menu require an active Internet connection.}%

\begin{figure}[ptb]%
\centering
\includegraphics[
natheight=5.119900in,
natwidth=9.439600in,
height=2.6907in,
width=4.9373in
]%
{fig-11.jpg}%
\caption{The \emph{R Markdown} tab, with \emph{Generate report} button.}%
\label{fig-11}%
\end{figure}


Each set of commands generated by the \textbf{R Commander} produces a block of
\textbf{R} commands in the \textbf{R Markdown} document.\footnote{Commands
that require direct user interaction, such as interactive point identification
in a graph, are suppressed in the \textbf{R Markdown} document. As well,
commands that generate errors are removed from the document.} These blocks are
delimited by \texttt{```\{r\}} at the start of each block and by \texttt{```}
(three back-ticks) at the end of the block. Even if you are unfamiliar with
\textbf{R} commands, you can see the relationship between commands and
resulting output in the \emph{Output} pane.

The \emph{R Markdown} tab is editable, and so you can modify and add to the
text in the tab. It is generally safe to type whatever explanatory text you
wish \emph{between} blocks of \textbf{R} code (see below). In general,
however, unless you know what you're doing, you should not modify \textbf{R}
code blocks in the \textbf{R Markdown} document or add your own code blocks.
You can, however, remove entire blocks of code, as long as subsequent blocks
don't depend on them \ --- and you will likely want to remove command blocks
that produce unwanted output. Command blocks that produce errors are removed
automatically. You can remove the most recent command block by selecting
\emph{Remove last Markdown Command Block} from the \emph{Edit} menu, or by
right-clicking in the \emph{R Markdown} tab and selecting \emph{Remove last
Markdown Command Block} from the context menu. The first code block (which
begins \texttt{```\{r echo=FALSE\}}) sets some options for the software from the \textbf{knitr} package (Xie,
2013) that is used to process the \textbf{R Markdown} text, and this block
should not normally be modified.

With some lines elided (indicated by \texttt{. . .}), here is the \textbf{R
Markdown} document produced for the current session:
\begin{verbatim}
---
title: "Replace with Main Title"
author: "Your Name"
date: "AUTOMATIC"
---


```{r echo=FALSE, message=FALSE}
# include this code chunk as-is to set options
knitr::opts_chunk$set(comment=NA, prompt=TRUE)
library(Rcmdr)
library(car)
library(RcmdrMisc)
```

```{r}
Nations <- read.table("C:/R/R-3.0.1patched/library/Rcmdr/etc/Nations.txt",
  header=TRUE, sep="", na.strings="NA", dec=".", strip.white=TRUE)
```

. . .


```{r}
data(Prestige, package="car")
```

. . .

Let us regress occupational prestige on the education and income levels of the occupations,
transforming income to linearize its relationship to prestige:

```{r}
LinearModel.1 <- lm(prestige ~ (education + log(income))*type,
  data=Prestige)
summary(LinearModel.1)
```

```{r}
Anova(LinearModel.1, type="II")
```

\end{verbatim}

\noindent It is probably unnecessary to explain that you would normally
replace \textquotedblleft\texttt{Your\ Name}\textquotedblright\ with your
name, and replace \textquotedblleft\texttt{Replace\ with\ Main\ Title}%
\textquotedblright\ with the title of the report that you want to create.
Perhaps less obviously, you can type arbitrary explanatory text before the
first command block, \emph{in between} \textbf{R} code blocks --- that is,
between the terminating \texttt{```} of one block and starting \texttt{```\{r\}}
of the next --- and after the last command block. You can take advantage of
the simple markup provided by \textbf{R Markdown}; for example, text enclosed
in asterisks (e.g., \texttt{*this is important*}) will be set in
\textit{italic} type. To illustrate, we added the text \textquotedblleft%
\texttt{Let us regress occupational prestige ...} \textquotedblright%
\ immediately before the \textbf{R} command block performing the regression.

Once you have finished editing the \textbf{R Markdown} document, you can
generate a report from it in the form of an HTML document (web page), \textbf{Word} document,
or PDF file by pressing the \emph{Generate report}
button below the \emph{R Markdown} tab.\footnote{To generate \textbf{Word} documents you must
install \textbf{Pandoc}; to generate PDF files, you'll additionally
need \LaTeX. To install this optional auxiliary software, use the \emph{Tools} menu.}
An HTML report should open in your web browser, and a PDF report in your PDF viewer.
\textbf{Word} files must be opened manually, but can be edited further.
The \textbf{R Markdown} document can be saved via the \emph{File} menu.

You can also, and more conveniently, open a separate and larger editor window to edit the \textbf{R
Markdown} document (see Figure \ref{fig-12}): Select \emph{Edit R Markdown
document} from the \textbf{R Commander} \emph{Edit} menu; right-click in the
\emph{R Markdown} tab and select \emph{Edit R Markdown document} from the
context menu; or press the key-combination \emph{Control-E} when the cursor is
in the \emph{R Markdown} tab. The editor supports the usual \emph{Edit}-menu
and right-click context-menu commands, and also allows you to compile the
\emph{R Markdown} document into an HTML report. Clicking the \emph{OK} button
in the editor saves your edits to the \emph{R Markdown} tab, and clicking
\emph{Cancel} discards your edits. Below the menu in the editor, there is a
largely self-explanatory toolbar with various buttons; if you hover the mouse
over a button, a \textquotedblleft tool-tip\textquotedblright\ will display.%

\begin{figure}[ptb]%
\centering
\includegraphics[
natheight=10.306300in,
natwidth=9.826400in,
height=5.3873in,
width=5.1374in
]%
{fig-12.jpg}%
\caption{The \textbf{R Markdown} document editor.}%
\label{fig-12}%
\end{figure}


\subsection{Saving and Printing Output}

You can also save text output directly from the \emph{File} menu in the
\textbf{R Commander;} likewise you can save or print a graph from the
\emph{File} menu in an \textbf{R} \emph{Graphics Device} window. If you prefer
not to use the \emph{R Markdown} tab, you can collect the text output and
graphs that you want to keep in a word-processor document. In this manner, you
can intersperse \textbf{R} output with your typed notes and explanations. This
procedure, however, has the disadvantage that it is not directly reproducible,
while an \textbf{R Markdown} document can subsequently be executed to
reproduce your analysis, possibly with modifications.

Open a word processor such as \textbf{Word},\textbf{\ OpenOffice Writer}, or
even \textbf{Windows} \textbf{WordPad}. To copy text from the \emph{Output}
pane, block the text with the mouse, select \emph{Copy} from the \emph{Edit}
menu (or press the key combination \emph{Ctrl-c}, or right-click in the pane
and select \emph{Copy} from the context menu), and then paste the text into
the word-processor document via \emph{Edit }$\longrightarrow$\emph{\ Paste}
(or \emph{Ctrl-v }or right-click \emph{Paste}), as you would for any
\textbf{Windows} application. One point worth mentioning is that you should
use a monospaced (\textquotedblleft\texttt{typewriter}\textquotedblright)
font, such as \emph{Courier New}, for text output from \textbf{R}; otherwise
the output will not line up neatly.

Likewise, to copy a graph, select \emph{File }$\longrightarrow$\emph{\ Copy to
the clipboard }$\longrightarrow$\emph{\ as a Metafile} from the \textbf{R}
\emph{Graphics Device} menus; then paste the graph into the word-processor
document via \emph{Edit }$\longrightarrow$\emph{\ Paste} (or \emph{Ctrl-v} or
right-click \emph{Paste}). Alternatively, you can use \emph{Ctrl-w} to copy
the graph from the \textbf{R} \emph{Graphics Device}, or right-click on the
graph to bring up a context menu, from which you can select \emph{Copy as
metafile}.\footnote{As you will see when you examine these menus, you can save
graphs in a variety of formats, and to files as well as to the clipboard. The
procedure suggested here is straightforward, however, and generally results in
high-quality graphs. Once again, this description applies to \textbf{Windows}
systems.} At the end of your \textbf{R} session, you can save or print the
document that you have created, providing an annotated record of your work.

Alternative routes to saving text and graphical output may be found
respectively under the \textbf{R Commander }\emph{File} and \emph{Graphs
}$\longrightarrow$\emph{\ Save graph to file} menus. Saving the \textbf{R
Commander} \emph{Script} tab, via \emph{File }$\longrightarrow$\emph{\ Save
script}, allows you to reproduce your work on a future occasion.

\subsection{Entering Commands in the Script Tab}

The \emph{R Script} tab provides a simple facility for editing, entering, and
executing commands. Commands generated by the \textbf{R Commander} appear in
the \emph{Script} tab, and you can type and edit commands in the tab more or
less as in any editor. The \textbf{R Commander} does not provide a true
\textquotedblleft console\textquotedblright\ for \textbf{R}, however, and the
\emph{Script} tab has some limitations. For example, all lines of a multiline
command must be submitted simultaneously for execution. For serious \textbf{R}
programming, it is preferable to use the script editors provided by the
\textbf{Windows} and \textbf{Mac OS X} versions of \textbf{R}, or --- even
better --- a programming editor or interactive development environment, such
as \textbf{RStudio} $<$\texttt{www.rstudio.org}$>$.\footnote{
The \textbf{R Commander} will run under \textbf{RStudio}, in which
case by default \textbf{R Commander} output and messages are directed to the
\textbf{R} console within \textbf{RStudio}, but there are some issues, such as
partial incompatibility with the \textbf{RStudio} \emph{Plot} and \emph{Help}
tabs.}

\subsection{Using R Commander Plug-ins\label{sec-plugins}}

\textbf{R Commander} plug-ins are \textbf{R} packages that add to the
capabilities of the \textbf{R Commander}. Many such plug-ins are currently
available on CRAN, and may be downloaded and installed in the normal manner.
Plug-ins typically add menus or menu items and associated dialog boxes to the
\textbf{R Commander}. They may also modify or remove existing menu items or
dialogs. A properly programmed \textbf{R Commander} plug-in can be loaded
either directly, in which case the \textbf{R Commander }loads along with the
plug-in, or from the \textbf{R Commander} via \emph{Tools} $\longrightarrow$
\emph{Load Rcmdr plug-in(s)...} . In the latter event, the \textbf{R
Commander} will restart to activate the plug-in. It is possible to use several
plug-ins simultaneously, but plug-ins may also conflict with each other. For
example, one plug-in may remove a menu to which another plug-in tries to add a
menu item.

\subsection{Terminating the R Session}

There are several ways to terminate your session. For example, you can select
\emph{File }$\longrightarrow$\emph{\ Exit }$\longrightarrow$\emph{\ From
Commander and R} from the \textbf{R Commander }menus. You will be asked to
confirm, and then asked whether you want to save the contents of the \emph{R
Script}, \emph{Output}, and \emph{R Markdow}n windows. Likewise, you can
select \emph{File }$\longrightarrow$\emph{\ Exit} from the \emph{R Console};
in this case, you will be asked whether you want to save the \textbf{R}
workspace (i.e., the data that \textbf{R} keeps in memory); you would normally
answer \emph{No}.

%\bibliographystyle{apa}
\bibliography{Rcmdr-Manual}



\end{document}
